% Options for packages loaded elsewhere
\PassOptionsToPackage{unicode}{hyperref}
\PassOptionsToPackage{hyphens}{url}
%
\documentclass[
]{book}
\usepackage{amsmath,amssymb}
\usepackage{lmodern}
\usepackage{ifxetex,ifluatex}
\ifnum 0\ifxetex 1\fi\ifluatex 1\fi=0 % if pdftex
  \usepackage[T1]{fontenc}
  \usepackage[utf8]{inputenc}
  \usepackage{textcomp} % provide euro and other symbols
\else % if luatex or xetex
  \usepackage{unicode-math}
  \defaultfontfeatures{Scale=MatchLowercase}
  \defaultfontfeatures[\rmfamily]{Ligatures=TeX,Scale=1}
\fi
% Use upquote if available, for straight quotes in verbatim environments
\IfFileExists{upquote.sty}{\usepackage{upquote}}{}
\IfFileExists{microtype.sty}{% use microtype if available
  \usepackage[]{microtype}
  \UseMicrotypeSet[protrusion]{basicmath} % disable protrusion for tt fonts
}{}
\makeatletter
\@ifundefined{KOMAClassName}{% if non-KOMA class
  \IfFileExists{parskip.sty}{%
    \usepackage{parskip}
  }{% else
    \setlength{\parindent}{0pt}
    \setlength{\parskip}{6pt plus 2pt minus 1pt}}
}{% if KOMA class
  \KOMAoptions{parskip=half}}
\makeatother
\usepackage{xcolor}
\IfFileExists{xurl.sty}{\usepackage{xurl}}{} % add URL line breaks if available
\IfFileExists{bookmark.sty}{\usepackage{bookmark}}{\usepackage{hyperref}}
\hypersetup{
  pdftitle={CC Primer: Compute Canada for geoscientists in a rush},
  pdfauthor={Ricardo Barros Lourenco},
  hidelinks,
  pdfcreator={LaTeX via pandoc}}
\urlstyle{same} % disable monospaced font for URLs
\usepackage{color}
\usepackage{fancyvrb}
\newcommand{\VerbBar}{|}
\newcommand{\VERB}{\Verb[commandchars=\\\{\}]}
\DefineVerbatimEnvironment{Highlighting}{Verbatim}{commandchars=\\\{\}}
% Add ',fontsize=\small' for more characters per line
\usepackage{framed}
\definecolor{shadecolor}{RGB}{248,248,248}
\newenvironment{Shaded}{\begin{snugshade}}{\end{snugshade}}
\newcommand{\AlertTok}[1]{\textcolor[rgb]{0.94,0.16,0.16}{#1}}
\newcommand{\AnnotationTok}[1]{\textcolor[rgb]{0.56,0.35,0.01}{\textbf{\textit{#1}}}}
\newcommand{\AttributeTok}[1]{\textcolor[rgb]{0.77,0.63,0.00}{#1}}
\newcommand{\BaseNTok}[1]{\textcolor[rgb]{0.00,0.00,0.81}{#1}}
\newcommand{\BuiltInTok}[1]{#1}
\newcommand{\CharTok}[1]{\textcolor[rgb]{0.31,0.60,0.02}{#1}}
\newcommand{\CommentTok}[1]{\textcolor[rgb]{0.56,0.35,0.01}{\textit{#1}}}
\newcommand{\CommentVarTok}[1]{\textcolor[rgb]{0.56,0.35,0.01}{\textbf{\textit{#1}}}}
\newcommand{\ConstantTok}[1]{\textcolor[rgb]{0.00,0.00,0.00}{#1}}
\newcommand{\ControlFlowTok}[1]{\textcolor[rgb]{0.13,0.29,0.53}{\textbf{#1}}}
\newcommand{\DataTypeTok}[1]{\textcolor[rgb]{0.13,0.29,0.53}{#1}}
\newcommand{\DecValTok}[1]{\textcolor[rgb]{0.00,0.00,0.81}{#1}}
\newcommand{\DocumentationTok}[1]{\textcolor[rgb]{0.56,0.35,0.01}{\textbf{\textit{#1}}}}
\newcommand{\ErrorTok}[1]{\textcolor[rgb]{0.64,0.00,0.00}{\textbf{#1}}}
\newcommand{\ExtensionTok}[1]{#1}
\newcommand{\FloatTok}[1]{\textcolor[rgb]{0.00,0.00,0.81}{#1}}
\newcommand{\FunctionTok}[1]{\textcolor[rgb]{0.00,0.00,0.00}{#1}}
\newcommand{\ImportTok}[1]{#1}
\newcommand{\InformationTok}[1]{\textcolor[rgb]{0.56,0.35,0.01}{\textbf{\textit{#1}}}}
\newcommand{\KeywordTok}[1]{\textcolor[rgb]{0.13,0.29,0.53}{\textbf{#1}}}
\newcommand{\NormalTok}[1]{#1}
\newcommand{\OperatorTok}[1]{\textcolor[rgb]{0.81,0.36,0.00}{\textbf{#1}}}
\newcommand{\OtherTok}[1]{\textcolor[rgb]{0.56,0.35,0.01}{#1}}
\newcommand{\PreprocessorTok}[1]{\textcolor[rgb]{0.56,0.35,0.01}{\textit{#1}}}
\newcommand{\RegionMarkerTok}[1]{#1}
\newcommand{\SpecialCharTok}[1]{\textcolor[rgb]{0.00,0.00,0.00}{#1}}
\newcommand{\SpecialStringTok}[1]{\textcolor[rgb]{0.31,0.60,0.02}{#1}}
\newcommand{\StringTok}[1]{\textcolor[rgb]{0.31,0.60,0.02}{#1}}
\newcommand{\VariableTok}[1]{\textcolor[rgb]{0.00,0.00,0.00}{#1}}
\newcommand{\VerbatimStringTok}[1]{\textcolor[rgb]{0.31,0.60,0.02}{#1}}
\newcommand{\WarningTok}[1]{\textcolor[rgb]{0.56,0.35,0.01}{\textbf{\textit{#1}}}}
\usepackage{longtable,booktabs,array}
\usepackage{calc} % for calculating minipage widths
% Correct order of tables after \paragraph or \subparagraph
\usepackage{etoolbox}
\makeatletter
\patchcmd\longtable{\par}{\if@noskipsec\mbox{}\fi\par}{}{}
\makeatother
% Allow footnotes in longtable head/foot
\IfFileExists{footnotehyper.sty}{\usepackage{footnotehyper}}{\usepackage{footnote}}
\makesavenoteenv{longtable}
\usepackage{graphicx}
\makeatletter
\def\maxwidth{\ifdim\Gin@nat@width>\linewidth\linewidth\else\Gin@nat@width\fi}
\def\maxheight{\ifdim\Gin@nat@height>\textheight\textheight\else\Gin@nat@height\fi}
\makeatother
% Scale images if necessary, so that they will not overflow the page
% margins by default, and it is still possible to overwrite the defaults
% using explicit options in \includegraphics[width, height, ...]{}
\setkeys{Gin}{width=\maxwidth,height=\maxheight,keepaspectratio}
% Set default figure placement to htbp
\makeatletter
\def\fps@figure{htbp}
\makeatother
\setlength{\emergencystretch}{3em} % prevent overfull lines
\providecommand{\tightlist}{%
  \setlength{\itemsep}{0pt}\setlength{\parskip}{0pt}}
\setcounter{secnumdepth}{5}
\usepackage{booktabs}
\ifluatex
  \usepackage{selnolig}  % disable illegal ligatures
\fi
\usepackage[]{natbib}
\bibliographystyle{plainnat}

\title{CC Primer: Compute Canada for geoscientists in a rush}
\author{Ricardo Barros Lourenco}
\date{Last revision: 2022-02-18}

\begin{document}
\maketitle

{
\setcounter{tocdepth}{1}
\tableofcontents
}
\hypertarget{front-matter}{%
\chapter{Front Matter}\label{front-matter}}

\hypertarget{copyright}{%
\section{Copyright}\label{copyright}}

Otherwise stated in the text, this content is licensed under a \href{http://creativecommons.org/licenses/by-nc-sa/4.0/}{Creative Commons Attribution-NonCommercial-ShareAlike 4.0 International License}.

\hypertarget{how-to-cite-this-work}{%
\subsection{How to cite this work}\label{how-to-cite-this-work}}

Please use this BibTeX fragment:

\begin{Shaded}
\begin{Highlighting}[]
\SpecialCharTok{@}\NormalTok{book\{BarrosLourenco2022,}
\NormalTok{  title     }\OtherTok{=} \StringTok{"CC Primer: Compute Canada for geoscientists in a rush"}\NormalTok{,}
\NormalTok{  author    }\OtherTok{=} \StringTok{"Barros Lourenco, Ricardo"}\NormalTok{,}
\NormalTok{  year      }\OtherTok{=} \DecValTok{2022}\NormalTok{,}
\NormalTok{  doi       }\OtherTok{=} \StringTok{"10.5281/zenodo.5937906"}
\NormalTok{  note      }\OtherTok{=}\NormalTok{ \{\textbackslash{}url\{https}\SpecialCharTok{:}\ErrorTok{//}\NormalTok{ricardobarroslourenco.github.io}\SpecialCharTok{/}\NormalTok{CCprimer}\SpecialCharTok{/}\NormalTok{\}(visited yyyy}\SpecialCharTok{{-}}\NormalTok{mm}\SpecialCharTok{{-}}\NormalTok{dd)\}}
\NormalTok{\}}
\end{Highlighting}
\end{Shaded}

Obs.: Note that the DOI is of a previous version (10.5281/zenodo.5937906) when
comapred to the badge displayed on this website, because it relates to all versions
of this document, rather than a especific one.

\hypertarget{cronology}{%
\section{Cronology}\label{cronology}}

2022-01-31 - Beginning of v0.1. (This version is quite unstable, and versioning
during this stage will be limited).

\hypertarget{version-control-systems}{%
\chapter{Version Control Systems}\label{version-control-systems}}

\emph{Version Control Systems} \citet{wiki_version_control} (VCS - also named as \emph{Revision Control},
\emph{Source Code Control}, \emph{Source Code Management}) are Computer Systems
traditionally used to manage the complexities of software development, majorly
Computer Systems involving multiple software modules, and large development teams.

Such systems help to organize such development, especially when collaboration of
multiple developers is done, and several times people are doing code changes in
a same, or close, part of the code, which may break it, or even generate
unexpected results that would not be traceable.

\hypertarget{vcs-local-repository-vs.-remote-repository}{%
\section{VCS: Local repository vs.~Remote repository}\label{vcs-local-repository-vs.-remote-repository}}

Common VCS's (ex.: \href{https://en.wikipedia.org/wiki/Concurrent_Versions_System}{CVS},
\href{https://en.wikipedia.org/wiki/Concurrent_Versions_System}{SVN} and
\href{https://en.wikipedia.org/wiki/Git}{Git}) always are client-server systems.

The VCS client is not merely a means to access a central repository hosted
at the VCS server, but actually is part of running checks and balances that are
necessary when adding code to the central repository. A situation that demands that
is when multiple users are doing a change in a same piece of code, and checking
it only at the central repository may complicate more a situation already
complicated (simultaneous contribution). Therefore, it hosts, locally
a local copy of the codebase, which reflects a view of that codebase in time
(more preciselly when that local user retrieved a copy from the central repository
for editing).

Once the changes are done, the user saves a local copy of these at the local repository,
and if these meet the criteria for changes (and this may vary a lot among systems),
the user is able to persist this change at the local repository, as a new version of
the codebase. This process is known as \emph{commit} (in this case, a local one).

\hypertarget{git}{%
\section{Git}\label{git}}

\href{https://git-scm.com/}{Git} is a contemporary VCS that is used as the backbone
of operations run on GitHub. So, when planning to operate with GitHub, it is
almost certain that you will need to install a Git client on your machine.

We can say \emph{almost} because GitHub provides an own client,
\href{https://desktop.github.com/}{GitHub Desktop} which provides a Graphical User
Interface(GUI) to a git client.

We will not cover such GUI usage since on Compute Canada you will not have access
to GitHub Desktop on their machines, but only to
a regular Git client (that you can use to access either GitHub, or another Git
compliant server such as a private repository built with
\href{https://about.gitlab.com/}{GitLab}).

\hypertarget{installing-a-git-client-on-your-machine}{%
\subsection{Installing a git client on your machine}\label{installing-a-git-client-on-your-machine}}

\hypertarget{ubuntu-or-any-other-debian-based-environment}{%
\subsubsection{Ubuntu (or any other Debian-based environment)}\label{ubuntu-or-any-other-debian-based-environment}}

First update your Operating System (OS) repository indexes:

\begin{Shaded}
\begin{Highlighting}[]
\ExtensionTok{$}\NormalTok{ sudo apt{-}get update}
\end{Highlighting}
\end{Shaded}

Then proceed to install:

\begin{Shaded}
\begin{Highlighting}[]
\ExtensionTok{$}\NormalTok{ sudo apt{-}get install git}
\end{Highlighting}
\end{Shaded}

Note: If you use Ubuntu, git is already provided as a base repository on this
distribution. If you use another linux, and this does not work for you, please
\href{https://github.com/ricardobarroslourenco/CCprimer/issues}{open an issue} on
this project and use the label \emph{requests}, and I will try to solve and include
here for future reference.

\hypertarget{apple}{%
\subsubsection{Apple}\label{apple}}

Apple is an OS that not use repositories by default, and git is not provided in
the main set of software. So you would have three main options to install git:

\begin{itemize}
\tightlist
\item
  Install Xcode (\emph{Preferred}): Apple has a software development kit (SDK) named
  \href{https://developer.apple.com/xcode/}{XCode} which provides git among several other
  software for MacOS and iOS software development. The advantage of using XCode´s
  git distribution is that it comes adjusted for your OS and is supported by Apple,
  whcih avoids conflicts and security issues.

  \begin{itemize}
  \tightlist
  \item
    To install XCode (and git by consequence), open the App Store, and search
    XCode as a software developed by Apple, and install it. To test, open the
    Terminal App, and run \emph{git}. You should receive an output of basic
    description of commands available on git.
  \end{itemize}
\item
  Install Homebrew, and then install Git (if you are a Homebrew user, or uses a
  lot of *.nix software not supported by Apple it is useful):

  \begin{itemize}
  \tightlist
  \item
    \href{https://brew.sh/}{Download and install homebrew} from their website;
  \item
    Install git as:
  \end{itemize}

\begin{Shaded}
\begin{Highlighting}[]
\ExtensionTok{$}\NormalTok{ brew install git}
\end{Highlighting}
\end{Shaded}

  \begin{itemize}
  \tightlist
  \item
    In the terminal, run \emph{git} and you should receive an output of basic
    description of commands available on git.
  \end{itemize}
\item
  Install a standalone version of git (\emph{not recommended}): You can download and
  install from git´s website a standalone client (I will not be covering this
  approach, because git will not be updated via this kind a install, posing a
  security risk and install this at your own risk - and effort!)
\end{itemize}

\hypertarget{windows}{%
\subsubsection{Windows}\label{windows}}

Since Windows works with standalone installations, the install follows as this:

\begin{itemize}
\tightlist
\item
  Go to the Git website and \href{https://git-scm.com/download/win}{download} the
  latest client (download the standalone version, unless you have severe storage
  limitations on your machine);
\item
  Once downloaded, run the installer with admin permissions and follow the
  default installation;
\item
  When installed, open PowerShell (PS) (or the command prompt, if you do not
  have PS installed), and run \emph{git} and hit enter. You should receive an output of basic
  description of commands available on git.
\end{itemize}

\hypertarget{setting-up-your-local-git-client}{%
\subsection{Setting up your local git client}\label{setting-up-your-local-git-client}}

Once you have your git client up and running, you need to setup the access
credentials to connect to a git repository, such as GitHub or GitLab for example.

To do so, run on your bash terminal (or command-line if on windows):

\begin{Shaded}
\begin{Highlighting}[]
\ExtensionTok{$}\NormalTok{ git config }\AttributeTok{{-}{-}}\NormalTok{ global user.name }\StringTok{"Your full name"}
\end{Highlighting}
\end{Shaded}

Note: On quotes you need to specify a name (depending on the environment,
it should be your full name, or an alias, such as employee number, for example). This
command has a global scope, so all users on your machine would have the same setup.

Then you need to specify an e-mail address (if on GitHub, it is preferred that
is linked to your account):

\begin{Shaded}
\begin{Highlighting}[]
\ExtensionTok{$}\NormalTok{ git config }\AttributeTok{{-}{-}}\NormalTok{ global user.email }\StringTok{"Your e{-}mail address"}
\end{Highlighting}
\end{Shaded}

Then check, if all values were set:

\begin{Shaded}
\begin{Highlighting}[]
\ExtensionTok{$}\NormalTok{ git config }\AttributeTok{{-}{-}list}
\end{Highlighting}
\end{Shaded}

You should see an output that reflects those previously set name and e-mail values.

\hypertarget{github}{%
\section{GitHub}\label{github}}

GitHub is a Web collaborative Version Control service based on a Git platform. It was
created in 2008, and as of today is the largest repository of source code in the World.
Aside of VCS capabilities, it provides other computational services, such as
Continuous Integration (CI) and Continuous Deployment (CD), Web Page Hosting on GitHub
Pages, security solutions, a software marketplace, among many others. It was recently
acquired by Microsoft, as one of the largest transactions in the tech domain.

GitHub has several different \href{https://docs.github.com/en/get-started/learning-about-github/githubs-products\#about-githubs-products}{tiers of access},
from basic free accounts up to corporate ones.

A main advantage for academic users would be using the \href{https://education.github.com}{GitHub Education} which includes a free \href{https://docs.github.com/en/get-started/learning-about-github/githubs-products\#github-pro}{GitHub PRO}
account and several software for free (or services credit hours) while you are
enrolled in an academic institution. GitHub Education also have \href{https://education.github.com/benefits}{different tiers of users}
aiming students, teachers, and the academic institutions itselves. Once you
create a simple free GitHub account, you can return back to GitHub Education,
and request to be enrolled in the program.

\hypertarget{creating-an-account}{%
\subsection{Creating an account}\label{creating-an-account}}

To create an account is simple:

\begin{itemize}
\tightlist
\item
  Go to \href{https://github.com/pricing}{GitHub pricing page}, select the Free tier,
  and click \emph{Join for Free};
\item
  Just follow the prompts to create your personal account.
\item
  You should receive an e-mail from them on the registered e-mail account, to
  verify your identity. If you fail to do so, your account would be basically \href{https://docs.github.com/en/get-started/signing-up-for-github/verifying-your-email-address\#about-email-verification}{useless}.
\end{itemize}

Note: Since I have created my account some years ago, I am just using GitHub's
user documentation as reference. If things get complicated in this step, please
let me know, and I will expand here.

\hypertarget{insert-github-credentials-on-your-local-git}{%
\subsection{Insert GitHub credentials on your local Git}\label{insert-github-credentials-on-your-local-git}}

Once you have your GitHub account created, and verified, it is time to setup these
credentials on your local Git client. While logged on your GitHub account you should:

\begin{itemize}
\tightlist
\item
  Click on your \emph{name/avatar/photo at the upper right corner of the screen}, and
  then click on \emph{Settings};
\item
  Then click on \emph{Developer Settings} (it is the last item on the bottom of
  items at the left panel);
\item
  Now click on the bottom item, \emph{Personal access tokens};
\item
  Then, click on \emph{Generate new token}, you will be requested again for your
  password on this step;
\item
  Now you will be required to fill in some info:

  \begin{itemize}
  \tightlist
  \item
    \emph{Note}: This will be a name of this Token. I often create a Token per
    software+device I use, then someone can write ``Rstudio on Laptop'', to
    differentiate from ``Rstudio on laboratory desktop''.

    \begin{itemize}
    \tightlist
    \item
      Isolating tokens on different (software,device) pairs helps isolating
      accesses to your GitHub account, and is important to contain a security
      breach. If someone is able to fetch one of your tokens, you will know from
      which machine and which software on it came from.
    \end{itemize}
  \item
    \emph{Expiration}: You should define a expiration date for your token. Choosing
    a date is an open ended question, but ideally systems exposed to the Web, or
    multiple users such an HPC cluster, should be changed frequently.

    \begin{itemize}
    \tightlist
    \item
      \emph{Avoid at all means to use the No expiration} mode, because you never
      want to forget unattended keys of your GitHub (ex. You graduate, and your
      key is left on a machine that another student will use in the lab).
    \end{itemize}
  \item
    \emph{Select scopes}: Perhaps this is the trickiest setup. The definition of a
    Token is actually a definition of a \href{https://en.wikipedia.org/wiki/OAuth}{OAuth}
    Token. On GitHub, this implies on being able to select all options that a user
    has in terms of operations on the platform, and actually, narrowing down to
    what a user wants to grant permission for in such Token.

    \begin{itemize}
    \tightlist
    \item
      Note: It is tempting to grant \emph{all permissions} to a single Token, but the user
      should ask it that is really necessary. In one end, granting all permissions
      is too permissive, and being as problematic as having a token with no
      expiration data.
    \item
      To look into what each scope covers, please look into this
      \href{https://docs.github.com/en/developers/apps/building-oauth-apps/scopes-for-oauth-apps\#available-scopes}{page}.
    \end{itemize}
  \end{itemize}
\item
  Once you have generated your token, treat is as a password (even in terms of
  security/sharing/etc.). You should not persist a copy of it, but only use for your
  local git setup. Even if you loose access to it, you can always revoke that
  token, and create a new one.
\item
  To set your local git to access GitHub (remind that you need first to set \href{https://ricardobarroslourenco.github.io/CCprimer/version-control-systems.html\#setting-up-your-local-git-client}{global variables}),
  you just need to access GitHub with it. A simple way, that will be further
  explained in more detail would be making a local copy of a repository using \emph{git clone}:
\end{itemize}

\begin{Shaded}
\begin{Highlighting}[]
\ExtensionTok{$}\NormalTok{ git clone https://github.com/a\_very\_weird\_user\_name/a\_more\_weird\_user\_project\_name.git}
\end{Highlighting}
\end{Shaded}

A concrete example would be the source-code of this document:

\begin{Shaded}
\begin{Highlighting}[]
\ExtensionTok{$}\NormalTok{ git clone https://github.com/ricardobarroslourenco/CCprimer.git}
\end{Highlighting}
\end{Shaded}

\begin{itemize}
\tightlist
\item
  Once done, your local git should request:

  \begin{itemize}
  \tightlist
  \item
    User-name: The one you have set for your GitHub account;
  \item
    Password: Your token.
  \end{itemize}
\item
  After this, it should make a local copy of that cloned repository (more
  especifically this copy will be stored at the directory path you have run the
  clone command - in bash / terminal / command line).

  \begin{itemize}
  \tightlist
  \item
    Note: If you have not been requested a username/password, maybe you have already
    a GitHub account already set on your machine (so please note if any errors occur
    when trying to access GitHub with git, no username + password is required -
    if these occur, they will be output on your bash / terminal / command line)
  \end{itemize}
\item
  Finally, to persist the token on your local install, run:
\end{itemize}

\begin{Shaded}
\begin{Highlighting}[]
\ExtensionTok{$}\NormalTok{ git config }\AttributeTok{{-}{-}global}\NormalTok{ credential.helper cache}
\end{Highlighting}
\end{Shaded}

\begin{itemize}
\tightlist
\item
  If you need to clean-up the token(s) installed:
\end{itemize}

\begin{Shaded}
\begin{Highlighting}[]
\ExtensionTok{$}\NormalTok{ git config }\AttributeTok{{-}{-}global} \AttributeTok{{-}{-}unset}\NormalTok{ credential.helper}
\end{Highlighting}
\end{Shaded}

\hypertarget{a-combined-usage-of-git-github}{%
\section{A combined usage of Git-GitHub}\label{a-combined-usage-of-git-github}}

This section describes a usage of Git-GitHub intended for scientific usage. Therefore,
is important to remark that there is still no consolidated practice on such application,
considering that these tools were not meant for academic usage, but rather for a software
engineering one. Such application varies across research groups and scientific domains, and on a best effort, we will summarize some good practices and update as needed.

\hypertarget{creating-a-repository}{%
\subsection{Creating a repository}\label{creating-a-repository}}

A \emph{repository} is a collection of code, stored on a git server (in this case,
a GitHub Repository). As any intellectual task, you may split your information/code
as desired.

In CS, people often keep a same service (a part of a system) at a
same repository.

In academia, is a consensus that researchers tend to keep all the codebase used
in a publication at a same repository. Then some variation occurs across communities.
The GIScience community, more especifically the R GIScience community, developed a

Depending on how you set your environment, there are some options to crea

\hypertarget{retrieving-code-from-a-repository}{%
\subsection{Retrieving code from a repository}\label{retrieving-code-from-a-repository}}

\hypertarget{submitting-code-to-a-repository}{%
\subsection{Submitting code to a repository}\label{submitting-code-to-a-repository}}

\hypertarget{conciliating-conflicts}{%
\subsection{\texorpdfstring{Conciliating \emph{``conflicts''}}{Conciliating ``conflicts''}}\label{conciliating-conflicts}}

\hypertarget{git-merge}{%
\subsubsection{Git Merge}\label{git-merge}}

\hypertarget{git-rebase}{%
\subsubsection{Git Rebase}\label{git-rebase}}

\hypertarget{contributing-to-open-source-projects}{%
\subsection{Contributing to open-source projects}\label{contributing-to-open-source-projects}}

\hypertarget{creating-a-pull-request}{%
\subsubsection{Creating a pull request}\label{creating-a-pull-request}}

  \bibliography{book.bib,packages.bib}

\end{document}
