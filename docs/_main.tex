% Options for packages loaded elsewhere
\PassOptionsToPackage{unicode}{hyperref}
\PassOptionsToPackage{hyphens}{url}
%
\documentclass[
]{book}
\usepackage{amsmath,amssymb}
\usepackage{lmodern}
\usepackage{ifxetex,ifluatex}
\ifnum 0\ifxetex 1\fi\ifluatex 1\fi=0 % if pdftex
  \usepackage[T1]{fontenc}
  \usepackage[utf8]{inputenc}
  \usepackage{textcomp} % provide euro and other symbols
\else % if luatex or xetex
  \usepackage{unicode-math}
  \defaultfontfeatures{Scale=MatchLowercase}
  \defaultfontfeatures[\rmfamily]{Ligatures=TeX,Scale=1}
\fi
% Use upquote if available, for straight quotes in verbatim environments
\IfFileExists{upquote.sty}{\usepackage{upquote}}{}
\IfFileExists{microtype.sty}{% use microtype if available
  \usepackage[]{microtype}
  \UseMicrotypeSet[protrusion]{basicmath} % disable protrusion for tt fonts
}{}
\makeatletter
\@ifundefined{KOMAClassName}{% if non-KOMA class
  \IfFileExists{parskip.sty}{%
    \usepackage{parskip}
  }{% else
    \setlength{\parindent}{0pt}
    \setlength{\parskip}{6pt plus 2pt minus 1pt}}
}{% if KOMA class
  \KOMAoptions{parskip=half}}
\makeatother
\usepackage{xcolor}
\IfFileExists{xurl.sty}{\usepackage{xurl}}{} % add URL line breaks if available
\IfFileExists{bookmark.sty}{\usepackage{bookmark}}{\usepackage{hyperref}}
\hypersetup{
  pdftitle={CC Primer: Compute Canada for geoscientists in a rush},
  pdfauthor={Ricardo Barros Lourenco},
  hidelinks,
  pdfcreator={LaTeX via pandoc}}
\urlstyle{same} % disable monospaced font for URLs
\usepackage{color}
\usepackage{fancyvrb}
\newcommand{\VerbBar}{|}
\newcommand{\VERB}{\Verb[commandchars=\\\{\}]}
\DefineVerbatimEnvironment{Highlighting}{Verbatim}{commandchars=\\\{\}}
% Add ',fontsize=\small' for more characters per line
\usepackage{framed}
\definecolor{shadecolor}{RGB}{248,248,248}
\newenvironment{Shaded}{\begin{snugshade}}{\end{snugshade}}
\newcommand{\AlertTok}[1]{\textcolor[rgb]{0.94,0.16,0.16}{#1}}
\newcommand{\AnnotationTok}[1]{\textcolor[rgb]{0.56,0.35,0.01}{\textbf{\textit{#1}}}}
\newcommand{\AttributeTok}[1]{\textcolor[rgb]{0.77,0.63,0.00}{#1}}
\newcommand{\BaseNTok}[1]{\textcolor[rgb]{0.00,0.00,0.81}{#1}}
\newcommand{\BuiltInTok}[1]{#1}
\newcommand{\CharTok}[1]{\textcolor[rgb]{0.31,0.60,0.02}{#1}}
\newcommand{\CommentTok}[1]{\textcolor[rgb]{0.56,0.35,0.01}{\textit{#1}}}
\newcommand{\CommentVarTok}[1]{\textcolor[rgb]{0.56,0.35,0.01}{\textbf{\textit{#1}}}}
\newcommand{\ConstantTok}[1]{\textcolor[rgb]{0.00,0.00,0.00}{#1}}
\newcommand{\ControlFlowTok}[1]{\textcolor[rgb]{0.13,0.29,0.53}{\textbf{#1}}}
\newcommand{\DataTypeTok}[1]{\textcolor[rgb]{0.13,0.29,0.53}{#1}}
\newcommand{\DecValTok}[1]{\textcolor[rgb]{0.00,0.00,0.81}{#1}}
\newcommand{\DocumentationTok}[1]{\textcolor[rgb]{0.56,0.35,0.01}{\textbf{\textit{#1}}}}
\newcommand{\ErrorTok}[1]{\textcolor[rgb]{0.64,0.00,0.00}{\textbf{#1}}}
\newcommand{\ExtensionTok}[1]{#1}
\newcommand{\FloatTok}[1]{\textcolor[rgb]{0.00,0.00,0.81}{#1}}
\newcommand{\FunctionTok}[1]{\textcolor[rgb]{0.00,0.00,0.00}{#1}}
\newcommand{\ImportTok}[1]{#1}
\newcommand{\InformationTok}[1]{\textcolor[rgb]{0.56,0.35,0.01}{\textbf{\textit{#1}}}}
\newcommand{\KeywordTok}[1]{\textcolor[rgb]{0.13,0.29,0.53}{\textbf{#1}}}
\newcommand{\NormalTok}[1]{#1}
\newcommand{\OperatorTok}[1]{\textcolor[rgb]{0.81,0.36,0.00}{\textbf{#1}}}
\newcommand{\OtherTok}[1]{\textcolor[rgb]{0.56,0.35,0.01}{#1}}
\newcommand{\PreprocessorTok}[1]{\textcolor[rgb]{0.56,0.35,0.01}{\textit{#1}}}
\newcommand{\RegionMarkerTok}[1]{#1}
\newcommand{\SpecialCharTok}[1]{\textcolor[rgb]{0.00,0.00,0.00}{#1}}
\newcommand{\SpecialStringTok}[1]{\textcolor[rgb]{0.31,0.60,0.02}{#1}}
\newcommand{\StringTok}[1]{\textcolor[rgb]{0.31,0.60,0.02}{#1}}
\newcommand{\VariableTok}[1]{\textcolor[rgb]{0.00,0.00,0.00}{#1}}
\newcommand{\VerbatimStringTok}[1]{\textcolor[rgb]{0.31,0.60,0.02}{#1}}
\newcommand{\WarningTok}[1]{\textcolor[rgb]{0.56,0.35,0.01}{\textbf{\textit{#1}}}}
\usepackage{longtable,booktabs,array}
\usepackage{calc} % for calculating minipage widths
% Correct order of tables after \paragraph or \subparagraph
\usepackage{etoolbox}
\makeatletter
\patchcmd\longtable{\par}{\if@noskipsec\mbox{}\fi\par}{}{}
\makeatother
% Allow footnotes in longtable head/foot
\IfFileExists{footnotehyper.sty}{\usepackage{footnotehyper}}{\usepackage{footnote}}
\makesavenoteenv{longtable}
\usepackage{graphicx}
\makeatletter
\def\maxwidth{\ifdim\Gin@nat@width>\linewidth\linewidth\else\Gin@nat@width\fi}
\def\maxheight{\ifdim\Gin@nat@height>\textheight\textheight\else\Gin@nat@height\fi}
\makeatother
% Scale images if necessary, so that they will not overflow the page
% margins by default, and it is still possible to overwrite the defaults
% using explicit options in \includegraphics[width, height, ...]{}
\setkeys{Gin}{width=\maxwidth,height=\maxheight,keepaspectratio}
% Set default figure placement to htbp
\makeatletter
\def\fps@figure{htbp}
\makeatother
\setlength{\emergencystretch}{3em} % prevent overfull lines
\providecommand{\tightlist}{%
  \setlength{\itemsep}{0pt}\setlength{\parskip}{0pt}}
\setcounter{secnumdepth}{5}
\usepackage{booktabs}
\ifluatex
  \usepackage{selnolig}  % disable illegal ligatures
\fi
\usepackage[]{natbib}
\bibliographystyle{plainnat}

\title{CC Primer: Compute Canada for geoscientists in a rush}
\author{Ricardo Barros Lourenco}
\date{Last revision: 2022-02-01}

\begin{document}
\maketitle

{
\setcounter{tocdepth}{1}
\tableofcontents
}
\hypertarget{front-matter}{%
\chapter{Front Matter}\label{front-matter}}

\hypertarget{copyright}{%
\section{Copyright}\label{copyright}}

Otherwise stated in the text, this content is licensed under a \href{http://creativecommons.org/licenses/by-nc-sa/4.0/}{Creative Commons Attribution-NonCommercial-ShareAlike 4.0 International License}.

\hypertarget{how-to-cite-this-work}{%
\subsection{How to cite this work}\label{how-to-cite-this-work}}

Please use this BibTeX fragment:

\begin{Shaded}
\begin{Highlighting}[]
\SpecialCharTok{@}\NormalTok{book\{BarrosLourenco2022,}
\NormalTok{  title     }\OtherTok{=} \StringTok{"CC Primer: Compute Canada for geoscientists in a rush"}\NormalTok{,}
\NormalTok{  author    }\OtherTok{=} \StringTok{"Barros Lourenco, Ricardo"}\NormalTok{,}
\NormalTok{  year      }\OtherTok{=} \DecValTok{2022}\NormalTok{,}
\NormalTok{  doi       }\OtherTok{=} \StringTok{"10.5281/zenodo.5937906"}
\NormalTok{  note      }\OtherTok{=}\NormalTok{ \{\textbackslash{}url\{https}\SpecialCharTok{:}\ErrorTok{//}\NormalTok{ricardobarroslourenco.github.io}\SpecialCharTok{/}\NormalTok{CCprimer}\SpecialCharTok{/}\NormalTok{\}(visited yyyy}\SpecialCharTok{{-}}\NormalTok{mm}\SpecialCharTok{{-}}\NormalTok{dd)\}}
\NormalTok{\}}
\end{Highlighting}
\end{Shaded}

\hypertarget{epigraph}{%
\section{Epigraph}\label{epigraph}}

\begin{quote}
"Here's to the crazy ones.
The misfits.
The rebels.
The troublemakers.
The round pegs in the square holes.

The ones who see things differently.

They're not fond of rules.
And they have no respect for the status quo.

You can quote them, disagree with them, glorify or vilify them.
About the only thing you can't do is ignore them.

Because they change things.

They push the human race forward.

While some may see them as the crazy ones,
we see genius.

Because the people who are crazy enough to think
they can change the world, are the ones who do."

--- Apple Computer, \emph{``Think different''} campaign. TBWA-Chiat-Day.
\end{quote}

\hypertarget{foreword}{%
\section{Foreword}\label{foreword}}

\hypertarget{list-of-abbreviations-and-acronyms}{%
\section{List of abbreviations and acronyms}\label{list-of-abbreviations-and-acronyms}}

\hypertarget{cronology}{%
\section{Cronology}\label{cronology}}

2022-01-31 - Beginning of v0.1.

\hypertarget{version-control-systems}{%
\chapter{Version Control Systems}\label{version-control-systems}}

\emph{Version Control Systems} \citet{wiki_version_control} (VCS - also named as \emph{Revision Control},
\emph{Source Code Control}, \emph{Source Code Management}) are Computer Systems
traditionally used to manage the complexities of software development, majorly
Computer Systems involving multiple software modules, and large development teams.

\hypertarget{why-using-it-out-of-cs}{%
\section{Why using it (out of CS)?}\label{why-using-it-out-of-cs}}

\hypertarget{git}{%
\section{Git}\label{git}}

\hypertarget{installing-a-git-client-on-your-machine}{%
\subsection{Installing a git client on your machine}\label{installing-a-git-client-on-your-machine}}

\hypertarget{ubuntu-or-any-other-.nix-environment}{%
\subsubsection{Ubuntu (or any other *.nix environment)}\label{ubuntu-or-any-other-.nix-environment}}

\hypertarget{apple}{%
\subsubsection{Apple}\label{apple}}

\hypertarget{windows}{%
\subsubsection{Windows}\label{windows}}

\hypertarget{setting-up-your-local-git-client}{%
\subsection{Setting up your local git client}\label{setting-up-your-local-git-client}}

\hypertarget{github}{%
\section{GitHub}\label{github}}

\hypertarget{creating-an-account}{%
\subsection{Creating an account}\label{creating-an-account}}

\hypertarget{creating-a-repository}{%
\subsection{Creating a repository}\label{creating-a-repository}}

\hypertarget{retrieving-code-from-a-repository}{%
\subsection{Retrieving code from a repository}\label{retrieving-code-from-a-repository}}

\hypertarget{submitting-code-to-a-repository}{%
\subsection{Submitting code to a repository}\label{submitting-code-to-a-repository}}

  \bibliography{book.bib,packages.bib}

\end{document}
